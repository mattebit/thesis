\chapter{Related work}
In this chapter other tools and works that are related to mine will be discussed.

\section{Micro ID Gym}
Micro ID Gym (MIG) "aims to assist system administrators and testers in the deployment and pen-testing of IdM protocol instances" - \cite{micro_id_gym}, inside this tool, two pentesting tools can be found:
\begin{itemize}
    \item MIG - OAuth/OIDC \cite{claudio_grisenti}
    \item MIG - SAML SSO \cite{stefano_facchini}
\end{itemize}
They are both plugins for \Gls{burp}, which have as objective to test the two different protocols. These plugins execute a series of actions on a browser, checks the messages in the background, and then, they tell a result.
These two plugins are similar to mine, but they are specifically created and defined to test SSO protocols only, and the tests that they used are fixed and cannot be easily edited, if a new test has to be implemented, the plugin has to be recompiled.

\section{SSO Testing language and Plugin}
The preceding two tools of MIG have been improved by a work similar to mine, the one done by my colleague Wendy Barreto \cite{wendy_barreto} in her bachelor thesis at Università di Trento to test \Gls{OAuth} and \Gls{OIDC} SSO protocols with a custom test definition pattern. Her work aimed at fixing the problem of hard-coded tests in the plugins for SSO protocols testing. The previous MIG plugin had been improved by removing the staticity of the test, adding the possibility to customly define all the tests with the use of a JSON language.
The available test actions worked well, but there were some limitations on the possible actions, especially in the active tests. For example:
\begin{itemize}
    \item Limited oracle for the verification of active tests, having just the verification of the correct execution of the operation and a check for the string "error" on the last page of the browser
    \item The filtering of the message to check or edit for static tests is limited, only "Authorization grant message", "Response messages", "Request messages" and "All messages" are available
    \item Only regex are supported to search something in a message
    \item Unable to work over encoded parameters
    \item Impossibility of doing multiple operations on a single message
    \item Impossibility of saving a parameter and using it somewhere else
    \item Impossibility of using multiple sessions in a test
\end{itemize}
Some of which stated as future works in Wendy's thesis. A more complete list of differences between the two languages and tools can be found in Attachment \ref{attachment:languages_comparison}.

Previously in this thesis I said that I have taken part of this work as a base to start with mine, editing it to suit the needs and objectives that had been defined. The idea of a language that could be used for any type of test over HTTP was born when I used her plugin, which was limited to \Gls{OAuth} and \Gls{OIDC} tests. I wanted to enlarge the possible tests to be defined without a restriction on a specific protocol. 
One of the things that has been used is the interface of the plugin, that has been modified, adding buttons and tabs to deal with multiple \gls{session track}s and other added functionalities. Also, the automation of the \gls{session track} was taken and edited, it actually was already used in \cite{claudio_grisenti,stefano_facchini}.

