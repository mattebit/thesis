\chapter{Conclusions and Future Work}
\label{chap:Conclusions}
This work aimed to design and implement an automated tool that could execute security tests specified with a language, allowing the tester to use an already defined test suite, saving time, and ensuring that all the tests for a specific protocol are done. Firstly, a formal declarative specification for the pentesting strategies has been designed, searching for all the possible actions and checks that the tester could need during pentesting. Once this has been done, the implementation of the declarative specification has started, forming a language to declare all the pentesting strategies in the form of tests. The next step was to design and implement a tool that could be used to execute the tests. The tool was built as a plugin for Burp. It was able to communicate with its proxy to manage sessions and manipulate messages. The definition of an OAuth/OIDC test suite written in the language has started during this phase. This was done to test the tool during development and check its results' correctness. In the end, this test suite was a complete set of tests to test the OAuth/OIDC protocol.

Lastly, the tool and language are working, as can be seen in \cite{sofia_zanrosso}. In this case, all the necessary tests have been specified in the language and executed, resulting in the expected result. The tool admitted to specify and execute tests that were not possible with other software. Also, the tool worked during the OAuth/OIDC testing phase, giving the same result as other compared software in the same tests executed. These two use cases give testers a good set of tests to test OAuth/OIDC and SAML protocols in an automated manner. The initial objective of having an automated tool to help the testers to identify vulnerabilities in services implementations can be considered reached. 

The software has still to be tested more deeply. It was used just by me and in \cite{sofia_zanrosso}, this means that there could be some stability problems and some bugs not yet discovered. This could be fixed in future works, with the problems and the limitations introduced in Section \ref{sec:limitations}. Some enhancements that could be added to the work could be the extension of the message filtering part of the language with the use of machine learning, making it possible to define a more abstract filter that does not solely rely on the search of parameters or exact strings. The message saving part of the software could be extended having different file formats, such as ".cap", this way the intercepted messages could then be imported into other software.
Another thing is that the test definition language could be extended even for other uses, like for low-level networking protocols such as routing protocols, electronic trading protocols, IP protocols, and more; Allowing to search for known vulnerabilities in real-time or just by analyzing saved packets files. For example, a plugin for Wireshark software could be developed. 

%più conclusioni di future works. Ripercorrere tutto, problemi, cosa ho fatto per risolverli.



