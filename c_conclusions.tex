\chapter{Conclusions and Future Works}
\label{chap:Conclusions}
The objectives fixed at the start of this thesis can be considered reached, the new tool and language is fully working as can be seen in \cite{sofia_zanrosso}, still, it is not perfect in terms of stability and affidability, as a deep testing phase should be accomplished, but overall, the tool is functioning well.

For future works, the problems and the limitations seen in \ref{sec:limitations} should be solved. The message filtering part of the language could be extended by the use of an AI, making it possible to define a more abstract filter that does not solely rely on the search of parameters or exact strings. Another thing is that the test definition language could be extended even for other uses, like for low-level networking protocols such as routing protocols, electronic trading protocols, IP protocols, and so on. Making it possible to search for known vulnerabilities in real-time or just by analyzing saved packets files. For example, a plugin for Wireshark software could be developed.
% Save the messages in a .cap file instead of a txt
%




