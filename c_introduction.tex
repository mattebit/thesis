\chapter{Introduction}
\label{chap:introduction}
In the last years, we are seeing a constant transition from physical to virtual, this is the case of bank transactions, government documents, health care data, and almost anything that can be virtualized. This is a great step forward, the storage is optimized and more easily accessible, facilitating the cataloging of the documents. But there are also some big concerns about security of that data, as it is virtual, the access to it is regulated by some authentication protocols instead of a physical identification of the subject accessing the physical documents. This means that it has to be checked whether the person accessing the data is the one that is claiming to be. The three basic principles over which data security is based are confidentiality, integrity and availability of data, this principle is called the CIA triad \cite{cia_triad}.
If those requirements are not met, sensitive data could be made inaccessible or even at risk of access from unauthorized parties, as a result, critical services for health care could be inaccessible, this is what happened for example in Lazio \cite{lazio_hacker_0,lazio_hacker_1}, Italy, on August 2021, where, because of a RansomWare infecting the internal computer network of the region, and keeping it down for multiple days, the covid vaccines and other health-care services could not be booked. Moreover, a lot of data in the systems has been encrypted, and so, denied of any access. It has to be said that preventing RansomWares infections is not the aim of this thesis, but this is a good example of what the consequences of exploiting a vulnerability could be. 

In this thesis the considered services are browser-based web-services (from now on service) that rely on HTTP based security protocols. The developing has been done with a particular focus over Identity Management (IdM) protocols such as Security Assertion Markup Language and OAuth, but, without limiting the possible applications to these protocols.
To ensure that the CIA triad requirements are satisfied, there is a need to test all the implementations of the protocols used by the service, that are liable of the security of the data. The services that provide access to the data has to be designed properly and tested by a software engineer or a security tester (from now on tester), over the well-known vulnerabilities. The well-known vulnerabilities are vulnerabilities of the protocols that have been discovered over time, and that are known to exist, the service should be built in a way to avoid them, and should be tested over them. But, there could arise some problems, for instance, the tester should be qualified to test the protocols and should know how they work, this is not always the case, as software engineers are sometimes responsible for doing that. Another problem is that due to the fast-evolving nature of protocols, new vulnerabilities will eventually be found, and so, new tests will have to be defined or old tests would have to be edited. This is a lot of work to be done by the tester. The result is that some vulnerabilities could remain undiscovered, representing a weakness in the system. The work of this thesis propose a new automated testing software and a language to help the tester to test the web-protocols used in the services mentioned above. 

Some softwares to test specific protocols and vulnerabilities already exist \cite{wendy_barreto,claudio_grisenti}, but they are usually very specific and hardcoded in a way that the tests could not be easily edited by the tester. Other tools exist, but they are not usually automated. With this work, I wanted to create a software that could be used to test any type of browser-based HTTP protocol and that could specify the tests in an abstract way, allowing the editing and adding. In order to do this, a new language will be used to specify the tests, and a software will be implemented to execute them. The test results will be evaluated by an oracle, that is composed by a series of components that are responsible for telling if the test is passed or not.

In this thesis the design and implementation of this new software will be discussed.

\section{Contributions}
The contributions of this thesis are the following:
\begin{itemize}
    \item A language to specify tests for protocols over HTTP, that makes possible to customly specify the tests based on the user needs
    \item A methodology to execute the tests specified by the language
    \item Implementation of a dedicated tool to execute the tests defined by the language
    \item An use-case scenario of an IdM protocol
    \item Experimental analysis and a test specification of OAuth/OIDC protocol
\end{itemize}

\section{Structure of the thesis}
The thesis is structured as follows:
\begin{itemize}
    \item \textbf{Chapter \ref{chap:Background}, Background}: A brief introduction over the concepts and softwares used in the rest of the thesis
    \item \textbf{Chapter \ref{chap:Design}, Design of the testing language}: The design of the tool and the testing language
    \item \textbf{Chapter \ref{chap:Implementation}, Implementation}: The implementation of the tool and the testing language
    \item \textbf{Chapter \ref{chap:Use_cases}, Use cases}: Use cases of the work of this thesis
    \item \textbf{Chapter \ref{chap:Related_work}, Related work}: Other works related to this
    \item \textbf{Chapter \ref{chap:Conclusions}, Conclusions and future works}: Conclusions, results, and works that could be done in the future
\end{itemize}




