\chapter{Introduction}
\label{chap:introduction}
In the last years, we have seen a constant transition from physical to virtual. This is the case of bank transactions, government documents, health care data, and almost anything that can be virtualized. This is a significant step forward to access services and resources anytime and anywhere. However, there are also some concerns about the security of services and data, that being virtual, can be accessed by some authentication protocols instead of a physical identification of the subject accessing the physical documents. This means that it has to be checked whether the person accessing the service or data is the one that is claiming to be. The three basic principles over which data security is based are confidentiality, integrity, and data availability. This principle is called the CIA triad \cite{cia_triad}.
If those requirements are not met, sensitive data could be made inaccessible or at the risk of access from unauthorized parties. As a result, critical services for health care could be inaccessible, and this is what happened for example in Lazio \cite{lazio_hacker_0,lazio_hacker_1}, Italy, in August 2021, where, because of a RansomWare infecting the internal computer network of the region and keeping it down for multiple days, the covid vaccines and other health-care services could not be booked. Moreover, many data in the systems have been encrypted and denied access. The attack was possible due to a stolen set of privileged credentials used to log into the system. It is not clear how the credentials were stolen, maybe with a phishing attack or in another way. Even if this particular attack was not due to a vulnerability exploit, this is an excellent example of what the consequences could be. 

In this thesis, the considered services are web services (from now on service) that rely on browser-based security protocols. Browser-based protocols are those used to secure web services communications, authentications, and identification. The development has been done with a particular focus over Identity Management (IdM) protocols such as \gls{SAML} and \gls{OAuth}/OIDC but without limiting the possible applications to these protocols.
To ensure that the CIA triad requirements are satisfied, there is a need to test all the implementations of the protocols used by the service that is liable for the security of the data. The services that provide access to the data have to be appropriately designed and tested by a software engineer or a security tester (from now on tester) at least w.r.t. a set of well-known vulnerabilities. These are vulnerabilities of the protocols that have been discovered over time and that are known to exist. The service should be built to avoid them and should be tested against them. However, there could arise some problems, for instance, the tester should be qualified to test the protocols and should know how they work. This is not always the case, as software engineers are sometimes responsible for that. Another problem is that due to the fast-evolving nature of protocols, new vulnerabilities will eventually be found, and so, new tests will have to be defined, or old tests will have to be edited. This is much work to be done by the tester. The result is that some vulnerabilities could remain undiscovered, representing a weakness in the system. This thesis proposes an automated testing software and a language to help the tester to test the web protocols used in the services mentioned above. 

Some software to test specific protocols and vulnerabilities already exist \cite{wendy_barreto,claudio_grisenti}, but they are usually very specific and hard-coded in a way that the tester could not easily edit the tests. Other tools exist, but they are not usually automated. With this work, I wanted to create a software that could test any browser-based protocol and abstractly specify the tests, allowing editing over time. To do this, a language will be used to specify the tests, and a software will be implemented to execute them. The test results will be evaluated by an oracle, which is composed of a series of components that are responsible for telling if the test is passed or not.

\section{Contributions}
\label{sec:contributions}
The contributions of this thesis are the following:
\begin{itemize}
    \item \textit{Design of a declarative specification language for pentesting strategies}: The pentesting strategies are the set of actions done on messages exchanged by protocols to verify if the service implementing the protocol is vulnerable to known vulnerabilities. The objective was to design a language that could define the necessary pentesting strategies. The language supports all the possible actions needed by a pentester to edit the messages, for instance, remove, add, and edit to be done on parts of the message. To verify if the service is vulnerable, there are also dedicated actions in the language to check the content of the messages. For example, to verify that the service is not vulnerable to a tested vulnerability, there could be the need to check the value or the presence of a specific parameter in the content of a specific message. This is done using a specific action in the language. The language also manages the browsers, making it possible to work on multiple sessions at once and, for example, to reply messages from one browser session to another. The previously introduced language was then implemented using JSON as a base language. The goal was to have a software that could automatically execute the specified tests. To do that, a series of actions (among those specified by the language) would have to be specified and then executed in a browser. The messages exchanged by the browser needs to be intercepted, and the tests have to be executed over these intercepted messages to give a result. For executing the actions on the browser, there was an already working solution \cite{wendy_barreto, claudio_grisenti} that has been taken as a starting point. My contribution was to design the test execution and how the tests would interact with the browser and to extend the action execution to manage multiple sessions and new actions. 

    \item \textit{Design and Implementation of a tool to execute the pentesting strategies}: The previously introduced language to execute tests against browser-based protocols is implemented. The \gls{burp} Suite (from now on, Burp), a security testing software for web security testing, has been chosen as the platform to be used as the ground for the implementation; it is based on a browser and a proxy that intercepts the traffic from it. It admits the execution of plugins written in Java that use a set of APIs to interact with the proxy. The tool will then be integrated as a plugin for \gls{burp}. A dedicated interpreter has been built in the tool to parse the tests written in the language. The parsed test is executed with the primitive actions in the browser. The tool is able to intercept the traffic from the browser using the Burp's APIs. Then, based on the language actions, it manipulates the messages before sending them back to the proxy. The last step is the execution of the checks specified in the tests; an oracle has been implemented using the checks defined in every test to evaluate the tests' correctness and give a result.
    \item \textit{Specification of a test suite for OAuth and OIDC protocols}: A test suite containing the well-known vulnerabilities of OAuth and OIDC protocols have been defined using the language. All of the OAuth/OIDC tests defined in \cite{claudio_grisenti, wendy_barreto} were considered and defined with the language, and new tests were also introduced, given the new possibilities introduced with the language and the tool. Overall, 12 tests were translated, and 6 were added. This was done to validate the expressivity of the language and its functionality. The new defined tests were by prevalence actives, they could be added because of the new features introduced by the language, such as the new multi-session environment and the possibility of saving the values of parameters in variables.
    \item \textit{Experimental analysis of the OAuth and OIDC test suite}: The defined OAuth and OIDC test suite analysis is accomplished. It has been used during the development of the tool to validate its functioning, comparing the obtained results with the results of different tools and manually checking the results over multiple protocol implementations to check its validity. The experimental analysis has been conducted over a test implementation of the protocol and some real-world implementations.
\end{itemize}

\section{Structure of the thesis}
The thesis is structured as follows:
\begin{itemize}
    \item \textbf{Chapter \ref{chap:Background}}: A brief introduction over the concepts and software used in the rest of the thesis is provided
    \item \textbf{Chapter \ref{chap:Design}}: The design of the tool and the testing language is given
    \item \textbf{Chapter \ref{chap:Implementation}}: The implementation of the tool and the testing language is described
    \item \textbf{Chapter \ref{chap:Use_cases}}: Some examples of application of the software and the language described in this thesis are illustrated
    \item \textbf{Chapter \ref{chap:Related_work}}: Other works related to the one of this thesis are discussed
    \item \textbf{Chapter \ref{chap:Conclusions}}: Conclusions, results, and works that could be done in the future are briefly commented
\end{itemize}




