\chapter{Introduction}
In the last years, we are seeing a constant transition from physical to virtual, this is the case of bank transactions, government documents, health care data, and almost anything that can be virtualized. This is a great step forward, the storage is optimized and more easily accessible, facilitating the cataloging of the document. But there are also some big concerns about security of that data, as it is virtual, the access to it is regulated by some authentication protocols instead of a physical identification of the subject accessing the physical documents. This means that we have to be sure that the person accessing the data is the one that is claiming to be. The three basic principles over which data security is based are confidentiality, integrity and availability of data, this is called the CIA triad \cite{cia_triad}.
If those requirements are not met, sensitive data could be inaccessible or even at risk of access from unauthorized parties, as a result, critical services for health care could be inaccessible, this is what happend for example in Lazio \cite{lazio_hacker_0} \cite{lazio_hacker_1}, Italy on August 2021, where because of a RansomWare infecting the internal computer network of the region, and keeping it down for multiple days, the covid vaccines and other health-care services could not be booked. Moreover, a lot of data in the systems has been encrypted, and so, denied of any access. It has to be said that preventing RansomWares infections is not the scope of this thesis, but this is a good example of what the consequences of exploiting a vulnerability could be.
To ensure the this requirements are satisfied, there is a need to test all the implementations of the protocols and system that are liable of the security of the data. Due to the fast-evolving nature of protocols, new vulnerabilities will eventually be found, and so, new tests will have to be defined or old tests would have to be edited. This is a lot of work to be done by a security tester, and the problem is that not always the person testing the implementation is qualified to do it. The result is that some vulnerabilities could be remain undiscovered, representing a weakness in the system.

An automated tool to test implementations of security protocols using a well-known list of vulnerabilities to be tested is needed, this is what the work described in this thesis is all about. Some tools to test specific protocols and vulnerabilities already exist \cite{wendy_barreto}\cite{claudio_grisenti}, but they are usually very specific and hardcoded in a way that the tests could not be easily edited. With this work, I wanted to create a tool that could be used to test any type of protocol that ran over HTTP, and that could be define the tests in a very abstract way, allowing edit and addings. In order to do this, a new language will be used to define the tests, and a software will be implemented to executhe those tests.

In this thesis the design and implementation of this new tool will be discussed.

