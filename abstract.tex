\chapter*{Abstract} % senza numerazione

\addcontentsline{toc}{chapter}{Abstract} % da aggiungere comunque all'indice
% L'abstract deve essere un riassunto dell'introduzione, contesto, motivazione, contributi
This thesis deals with the work started in my internship at Fondazione Bruno Kessler (FBK) in the context of browser-based protocols (based on HTTP) security testing.
In the last years, there has been a constantly rising number of services transitioning from physical to virtual, such as health care and government services, also due to the Covid-19 pandemics. This was done to simplify some procedures and make them faster to be accomplished. However, some security and privacy concerns arise from this transition. A high quantity of sensitive data could be at risk of unwanted access. Software and security engineers have to implement the service's software and manage the data that the software is working with. They will rely on browser-based security protocols to ensure that security requirements such as user identity, the connection between the service and the user, and data integrity are met. 

To ensure that all these security requirements are respected during the implementation of the service, there is the need to test the implemented services against the well-known vulnerabilities of the protocols used. Usually, to do that, a software engineer or a security tester, which will be called tester below, will act like an attacker trying to see, edit, and deny access to the data in the services. The tester will use a software to intercept the HTTP messages (from now on messages) between the service and the tester's browser. It will be possible for the tester to edit or remove the messages, simulate an attack, and see if the service is vulnerable. The tester will have to gather all the well-known vulnerabilities related to the used protocols and test them manually. The pentesting is time-consuming and could be done improperly if the tester is not qualified to do it. A software is needed to automatically test the services against the well-known vulnerabilities, which could give a result of whether the services are vulnerable.

The work in this thesis is about the design and implementation of the automatic HTTP pentesting software (from now on tool) and its relative declarative specification language. The tool addresses the challenges highlighted above, such as the automation of the pentesting phase to improve its precision and the design of a language for the specification of the tests. Other software has been evaluated before the start of this work, some of them missed the automation part, others were limited in terms of the possible security tests (from now on tests) to be done. For these reasons, a tool and language have been designed and implemented. First, a declarative specification of pentesting strategies has been designed, simultaneously with the design of an approach to execute the pentesting strategies. Next, the implementation of a tool to execute the pentesting strategies has been done. To test the correctness of the implemented tool and the expressiveness of the proposed specification language, a test definition for \gls{OAuth}/OIDC, a web security protocol, has been written in the implemented language and used during development.

The implemented tool and language have proved to be working in two different use cases, with OAuth/OIDC and SAML. The results obtained by the tool have also been confirmed by comparison with the results of similar software; they have proved to be correct. Lastly, the objectives fixed at the beginning of the work were accomplished, the software and language created are valuable tools to help the tester discover vulnerabilities in services. The definition of an OAuth/OIDC and SAML test suite covering the well-known vulnerabilities is useful as the tester does not need to write them every time to test them. Depending on the service to test, what has to be changed are the automation actions to be executed during testing, as the web pages may be different, so the actions to be accomplished have to be different. Future work has been planned to extend the tool with other functionalities and support other protocols. 






