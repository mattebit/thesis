\chapter*{Abstract} % senza numerazione

\addcontentsline{toc}{chapter}{Abstract} % da aggiungere comunque all'indice
% L'abstract deve essere un riassunto dell'introduzione, contesto, motivazione, contributi
This thesis deals with the work started in my internship at Fondazione Bruno Kessler (FBK) in the context of HTTP browser-based protocol testing.
In the last years, there has been a constantly rising number of services transitioning from physical to virtual, such as health care services and government services. This was done to simplify some procedures, and so, making them faster to be accomplished. However, some concerns arise from this transition. A high quantity of sensitive data could be at risk of unwanted access. Software and security engineers have to implement the service's software and manage the data that the software is working with. They will rely on browser-based HTTP security protocols to assure the user identity, secure the connection between the service and the user, and assure data integrity. 

To assure all these security requirements are met in the implementation of the service, there is the need to test the implemented services against the well-known vulnerabilities of the protocols used. Usually, to do that, a software engineer or a security tester, which will be called tester, will act like an attacker trying to see, edit, and deny access to the data in the services. A software will intercept the HTTP messages between the service and the tester's browser. It will make it possible for the tester to edit or remove the messages, simulate an attack, and see if the service is vulnerable. The tester will have to gather all the well-known vulnerabilities related to the used protocols and test them manually. This is very time-consuming and could be done improperly if the tester is not qualified to do it. That is why a software able to automatically test the services against the well known-vulnerabilities and that could give a result of whether the services are vulnerable or not is needed. The advantages are that the group of people who could test the implemented services could be enlarged to people who are not qualified to do security testing.

The work in this thesis is about the design and implementation of the automatic HTTP pentesting software and its relative declarative specification language. Other software has been evaluated before deciding to start this work, some of them missed the automation part, others were limited in terms of the possible test to be done. For these reasons, a new software and language have been designed and implemented. First, a design of a declarative specification of pentesting strategies has been done with the contemporary design of an approach to execute the pentesting strategies. Next, the implementation of a tool to execute the pentesting strategies has been done. To test the correctness of the implemented tool, a test definition for \gls{OAuth}, a web security protocol, has been written in the implemented language and used during development.

The implemented software and language have proved to be working in two different use cases, an OAuth test suite and a SAML pentesting scenario. The results have also been confirmed by comparison with the results of similar software. In the end, the objectives fixed at the beginning of the work were accomplished, the software and language created are an useful tool to help the tester to test the services. The definition of an OAuth and SAML test suite comprehending the well-known vulnerabilities is useful as the tester doesn't need to write them to test them, just the automation actions have to be rewritten depending on the service to test. There is still some testing to be done to the software and some small things that could be improved, but overall it is working well.






