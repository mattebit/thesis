\chapter{Background} % senza numerazione
\label{Background}

\addcontentsline{toc}{chapter}{Sommario} % da aggiungere comunque all'indice

The idea of a pentesting tool used to test SSO implementations such as OAuth, OIDC and SAML was previously discussed by my colleagues Stefano Facchini \cite{stefano_faccini}, Claudio Grisenti and Wendy Barreto, which developed a plugin in Burp, that is a toolkit for web security testing.\\

\section{The old plugin}
The plugin work by executing a browser session and by doing some automatic actions defined in a track. All the messages received and sent by the browser are intercepted by the Burp's proxy. The plugin checks the messages and based on the test defined in the plugin tells if there is a vulnerability or not.
Starting from the first version of Stefano, to the last of Wendy, the plugin was improved. In the first two versions of Stefano and Claudio the plugin was static, in a way that ony the supported tests could be executed, with little settings to change. This version of the plugin worked well, but as said the tests could not be customized.

\section{Last plugin version}
Wendy improved the plugin by removing the staticity of the test, adding the possibility to customly define all of the tests by some specific JSON objects. \\
This is the last version of the plugin which i started working to. The plugin supported the definition of static and dynamic tests: static tests are tests where the messages are not edited, dynamic tests are tests where there could be an edit of one or more messages. The tests where working, but were very limited in the types of actions that could be defined. For example:
\begin{itemize}
    \item Limited oracle for the verification of active tests
    \item The filtering of which message to check or edit for static tests was limited: ( only "Authorization grant message", "Response messages", "Request messages" and "All messages")
    \item Only regex were supported to search something in a message
    \item Unable to work over encoded parameters
    \item Impossibility of doing multiple operations on a single message (verify)
    \item Impossibility of saving a parameter and using it somewhere else
\end{itemize}
Moreover, the language was thought with OIDC and OAuth in mind, other SSO protocols such as SAML could not be tested, because of the fact that SAML parameters are encoded, so editing them or verifying them is not possible.
 