\chapter{A more abstract Language}
\label{cha:intro}

I decided to start from the last version of the language and to completely rethink it, whith a more abstract approach, in a way that almost any type of protocol that uses HTTP could be tested. I started gathering all types of tests that could be wanted to be executed, searching for all the possible actions that could be done to a message.
\cite{dalal}

\section{Pellentesque habitant morbi tristique senectus}
\label{sec:context}

Lorem ipsum dolor sit amet, consectetur adipiscing elit. Donec sed nunc orci. Aliquam nec nisl vitae sapien pulvinar dictum quis non urna. Suspendisse at dui a erat aliquam vestibulum. Quisque ultrices pellentesque pellentesque. Pellentesque egestas quam sed blandit tempus. Sed congue nec risus posuere euismod. Maecenas ut lacus id mauris sagittis egestas a eu dui. Class aptent taciti sociosqu ad litora torquent per conubia nostra, per inceptos himenaeos. Pellentesque at ultrices tellus. Ut eu purus eget sem iaculis ultricies sed non lorem. Curabitur gravida dui eget ex vestibulum venenatis. Phasellus gravida tellus velit, non eleifend justo lobortis eget.
%\cite{ictbusiness}
%\cite{donoho}

\section{Nullam et justo vitae nisi}
\label{sec:problem}

Lorem ipsum dolor sit amet, consectetur adipiscing elit. Donec sed nunc orci. Aliquam nec nisl vitae sapien pulvinar dictum quis non urna. Suspendisse at dui a erat aliquam vestibulum. Quisque ultrices pellentesque pellentesque. Pellentesque egestas quam sed blandit tempus. Sed congue nec risus posuere euismod. Maecenas ut lacus id mauris sagittis egestas a eu dui. Class aptent taciti sociosqu ad litora torquent per conubia nostra, per inceptos himenaeos. Pellentesque at ultrices tellus. Ut eu purus eget sem iaculis ultricies sed non lorem. Curabitur gravida dui eget ex vestibulum venenatis. Phasellus gravida tellus velit, non eleifend justo lobortis eget.


