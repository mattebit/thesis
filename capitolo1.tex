\chapter{Changes to the language and plugin}
\label{cha:intro}

\section{A more abstract Language}
\label{sec:context}

I decided to start from the last version of the language and to completely rethink it, whith a more abstract approach, in a way that almost any type of protocol that uses HTTP could be tested. I started gathering all types of tests that could be wanted to be executed, searching for all the possible actions that could be done to a message.

\begin{center}
    \begin{tabular}{ |c|c|c| }
        \hline
        Action                                     & Old language         & New language                             \\
        \hline
        Custom message filtering                   & Only on active tests & supported                                \\
        Edit string                                & only by regex        & supported with regex and check construct \\
        Remove string                              & only by regex        & supported with regex and check construct \\
        Add string                                 & not supported        & supported                                \\
        Check parameter                            & only with regex      & with regex and check construct           \\
        Multiple checks in single message          & not supported        & supported                                \\
        Saving and reusing of strings and messages & not supported        & supported                                \\
        Multiple sessions in single test           & not supported        & supported                                \\
        Custom oracle definition                   & not supported        & by using regex and checks                \\

        \hline
    \end{tabular}
\end{center}