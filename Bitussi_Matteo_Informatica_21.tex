% formato FRONTE RETRO
\documentclass[epsfig,a4paper,11pt,titlepage,twoside,openany]{book}
\usepackage{epsfig}
\usepackage{plain}
\usepackage{setspace}
\usepackage[paperheight=29.7cm,paperwidth=21cm,outer=1.5cm,inner=2.5cm,top=2cm,bottom=2cm]{geometry} % per definizione layout
\usepackage{titlesec} % per formato custom dei titoli dei capitoli
\usepackage{graphicx}
\graphicspath{{./images/}}
\usepackage{listings}
\usepackage{xcolor}


\definecolor{codegreen}{rgb}{0,0.6,0}
\definecolor{codegray}{rgb}{0.5,0.5,0.5}
\definecolor{codepurple}{rgb}{0.58,0,0.82}
\definecolor{backcolour}{rgb}{0.95,0.95,0.92}

\lstdefinestyle{mystyle}{
    backgroundcolor=\color{backcolour},   
    commentstyle=\color{codegreen},
    keywordstyle=\color{magenta},
    numberstyle=\tiny\color{codegray},
    stringstyle=\color{codepurple},
    basicstyle=\ttfamily\footnotesize,
    breakatwhitespace=false,         
    breaklines=true,                 
    captionpos=b,                    
    keepspaces=true,                 
    numbers=left,                    
    numbersep=5pt,                  
    showspaces=false,                
    showstringspaces=false,
    showtabs=false,                  
    tabsize=2
}

\lstset{style=mystyle}

% JSON highlight definition
\definecolor{delim}{RGB}{20,105,176}
\definecolor{numb}{RGB}{106, 109, 32}
\definecolor{string}{rgb}{0.64,0.08,0.08}
\lstdefinelanguage{json}{
    numbers=left,
    numberstyle=\small,
    rulecolor=\color{black},
    showspaces=false,
    showtabs=false,
    breaklines=true,
    postbreak=\raisebox{0ex}[0ex][0ex]{\ensuremath{\color{gray}\hookrightarrow\space}},
    breakatwhitespace=true,
    basicstyle=\ttfamily\small,
    upquote=true,
    morestring=[b]",
    stringstyle=\color{string},
    literate=
     *{0}{{{\color{numb}0}}}{1}
      {1}{{{\color{numb}1}}}{1}
      {2}{{{\color{numb}2}}}{1}
      {3}{{{\color{numb}3}}}{1}
      {4}{{{\color{numb}4}}}{1}
      {5}{{{\color{numb}5}}}{1}
      {6}{{{\color{numb}6}}}{1}
      {7}{{{\color{numb}7}}}{1}
      {8}{{{\color{numb}8}}}{1}
      {9}{{{\color{numb}9}}}{1}
      {\{}{{{\color{delim}{\{}}}}{1}
      {\}}{{{\color{delim}{\}}}}}{1}
      {[}{{{\color{delim}{[}}}}{1}
      {]}{{{\color{delim}{]}}}}{1},
}

% supporto lettere accentate
%\usepackage[latin1]{inputenc} % per Windows;
\usepackage[utf8x]{inputenc} % per Linux (richiede il pacchetto unicode);
%\usepackage[applemac]{inputenc} % per Mac.

\singlespacing

\usepackage[english]{babel}

\begin{document}

% nessuna numerazione
\pagenumbering{gobble}
\pagestyle{plain}

\thispagestyle{empty}

\begin{center}
  \begin{figure}[h!]
    \centerline{\psfig{file=marchio_unitrento_colore_it_202002.eps,width=0.6\textwidth}}
  \end{figure}

  \vspace{2 cm} 

  \LARGE{Dipartimento di Ingegneria e Scienza dell’Informazione \\ Departement of Information Engineering and Computer Science \\}

  \vspace{1 cm} 
  \Large{Bachelor's Degree in \\ Computer Science}

  \vspace{2 cm} 
  \Large\textsc{Final dissertation\\} 
  \vspace{1 cm} 
  \Huge\textsc{Declarative Specification of Pentesting Strategies for Browser-based Security Protocols:
  the Case Studies of SAML and OAuth/OIDC\\}
  %\Large{\it{Implementation of an automatic testing language and tool to test HTTP browser based protocols}}


  \vspace{1 cm} 
  \begin{tabular*}{\textwidth}{ c @{\extracolsep{\fill}} c }
  \Large{Supervisor} & \Large{Student}\\
  \Large{Silvio Ranise}& \Large{Matteo Bitussi}\\
  \end{tabular*}

  \vspace{0.5 cm} 
  \begin{tabular*}{\textwidth}{ c @{\extracolsep{\fill}} c }
  \Large{Co-Supervisors} & \Large{}\\
  \Large{Andrea Bisegna}& \Large{}\\
  \Large{Roberto Carbone}& \Large{}\\
  \end{tabular*}

  \vspace{2 cm} 

  \Large{Academic year 2021/2022}
  
\end{center}



\clearpage

% Ringraziamenti
\thispagestyle{empty}

\begin{center}
  {\bf \Huge Ringraziamenti}
\end{center}

\vspace{4cm}


\emph{
%Thanks to Professor Ranise for the opportunity of working in the cybersecurity field with a real-world problem to solve.
%Thanks to Andrea and Roberto for the new things introduced to me and for the help with this work.
%Thanks to my parents for introducing me to new things since I was young, for always promoting my ideas and creativity, and for supporting me and my studies.
%Thanks to Alice, for the help and support during the long exam sessions and the writing of this thesis.
%Thanks to all the E-Agle TRT Racing Team, for the amazing experience, for the infinite amount of knowledge that it gave to me, for the fun and emotions of building a car from scratch and seeing it racing, and for the lateness in my graduation.
}



\clearpage
\pagestyle{plain} % nessuna intestazione e pie pagina con numero al centro


% inizio numerazione pagine in numeri arabi
\mainmatter

%%%%%%%%%%%%%%%%%%%%%%%%%%%%%%%%%%%%%%%%%%%%%%%%%%%%%%%%%%%%%%%%%%%%%%%%%%
%%%%%%%%%%%%%%%%%%%%%%%%%%%%%%%%%%%%%%%%%%%%%%%%%%%%%%%%%%%%%%%%%%%%%%%%%%
%% Nota
%%%%%%%%%%%%%%%%%%%%%%%%%%%%%%%%%%%%%%%%%%%%%%%%%%%%%%%%%%%%%%%%%%%%%%%%%%
%% Si ricorda che il numero massimo di facciate e' 30.
%% Nel conteggio delle facciate sono incluse 
%%   indice
%%   sommario
%%   capitoli
%% Dal conteggio delle facciate sono escluse
%%   frontespizio
%%   ringraziamenti
%%   allegati    
%%%%%%%%%%%%%%%%%%%%%%%%%%%%%%%%%%%%%%%%%%%%%%%%%%%%%%%%%%%%%%%%%%%%%%%%%%
%%%%%%%%%%%%%%%%%%%%%%%%%%%%%%%%%%%%%%%%%%%%%%%%%%%%%%%%%%%%%%%%%%%%%%%%%%

% indice
\tableofcontents
\clearpage

\chapter*{Abstract} % senza numerazione

\addcontentsline{toc}{chapter}{Abstract} % da aggiungere comunque all'indice
% L'abstract deve essere un riassunto dell'introduzione, contesto, motivazione, contributi
This thesis deals with the work started in my internship at Fondazione Bruno Kessler (FBK) in the context of browser-based protocols (based on HTTP) security testing.
In the last years, there has been a constantly rising number of services transitioning from physical to virtual, such as health care services and government services. This was done to simplify some procedures, and so, making them faster to be accomplished. However, some concerns arise from this transition. A high quantity of sensitive data could be at risk of unwanted access. Software and security engineers have to implement the service's software and manage the data that the software is working with. They will rely on browser-based security protocols to assure that security requirements such as the user identity, the connection between the service and the user, and data integrity are met. 

To ensure all these security requirements are respected during the implementation of the service, there is the need to test the implemented services against the well-known vulnerabilities of the protocols used. Usually, to do that, a software engineer or a security tester, which will be called tester, will act like an attacker trying to see, edit, and deny access to the data in the services. The tester will use a software to intercept the HTTP messages (from now on messages) between the service and the tester's browser. It will be possible for the tester to edit or remove the messages, simulate an attack, and see if the service is vulnerable. The tester will have to gather all the well-known vulnerabilities related to the used protocols and test them manually. The pentesting is time-consuming and could be done improperly if the tester is not qualified to do it. A software is needed to automatically test the services against the well known-vulnerabilities, that could give a result of whether the services are vulnerable or not. The advantages are that the group of people who could test the implemented services could be enlarged to people who are not qualified to do security testing.

The work in this thesis is about the design and implementation of the automatic HTTP pentesting software (from now on tool) and its relative declarative specification language. Other software has been evaluated before the start of this work, some of them missed the automation part, others were limited in terms of the possible security tests (from now on tests) to be done. For these reasons, a tool and language have been designed and implemented. First, a declarative specification of pentesting strategies has been designed, simultaneously designing an approach to execute the pentesting strategies. Next, the implementation of a tool to execute the pentesting strategies has been done. To test the correctness of the implemented tool, a test definition for \gls{OAuth}/OIDC, a web security protocol, has been written in the implemented language and used during development.

The implemented tool and language have proved to be working in two different use cases, with OAuth/OIDC and SAML. The results obtained by the tool have also been confirmed by comparison with the results of similar software; they have proved to be correct. Lastly, the objectives fixed at the beginning of the work were accomplished, the software and language created are valuable tools to help the tester test the services. The definition of an OAuth/OIDC and SAML set of tests (test suite) comprehending the well-known vulnerabilities is useful as the tester does not need to write them to test them, just the automation actions have to be rewritten depending on the service to test. Some future work has been planned to extend the tool with other functionalities and support other protocols.







\clearpage

% gruppo per definizone di successione capitoli senza interruzione di pagina
\begingroup
% nessuna interruzione di pagina tra capitoli
% ridefinizione dei comandi di clear page
\renewcommand{\cleardoublepage}{}
\renewcommand{\clearpage}{}
% redefinizione del formato del titolo del capitolo
% da formato
%   Capitolo X
%   Titolo capitolo
% a formato
%   X   Titolo capitolo

\titleformat{\chapter}
{\normalfont\Huge\bfseries}{\thechapter}{1em}{}

\titlespacing*{\chapter}{0pt}{0.59in}{0.02in}
\titlespacing*{\section}{0pt}{0.20in}{0.02in}
\titlespacing*{\subsection}{0pt}{0.10in}{0.02in}

%%%%%%%%%%%%%%%%%%%%%%%%%%%%%%%%
% lista dei capitoli
%
% \input oppure \include
%
\chapter{Background}
The idea of a pentesting tool used to test SSO implementations such as OAuth, OIDC and SAML was previously discussed by my colleagues Stefano Facchini \cite{stefano_faccini}, Claudio Grisenti \cite{claudio_grisenti} and Wendy Barreto \cite{wendy_barreto}, which developed a plugin in Burp with the intent of automating the the testing of OIDC and OAuth protocols.

\section{Burp Suite}
Burp is one of the most used application security testing software for web security testing. It works by the use of a proxy server over which a browser redirect the traffic to. Burp has access to the proxy, it can sniff HTTP packets and can edit them. Burp also gives the possibility of creating custom plugins giving to the developers access to the java API.



\chapter{Design}
In the Design chapter I am going to talk about how the language and the plugin have been designed and how they work.

\section{The Language}
The idea was to think of a language that could implement all the possible actions which a security tester would be wanting to do on a multipart webapp test.\\
I had to decide how to write and define the actual tests, i thought i could define a proper language with a dedicated parser, but it was not worth the effort, as there are already some well-tested alternatives available. I found a great alternative: i used JSON as a base over which write the tests. It is a convinient way of defining gerarchical sturctures like tests could be.
The idea behind this language is that a specific message can be intercepted and checked or edited in some way, to do this we define various types
The gerarchical structure and the details of the language will be discussed in the next charapter.

\subsection{Test example: PKCE Downgrade}
I want to introduce the language with an example. Due to its complexity, having a real example before the explanation of all its components could be helpful to understand their use.
The implemented test has as objective to test an OAuth vulnerability where removing the parameter "code\_challenge" from the url of an authorization request message will be downgrading the authentication proces in a way that PKCE will not be used if the service is vulnerable.


\begin{lstlisting}[language=json]
{
    "test suite": {
        "name": "OAuth Active tests",
        "description": "A series of tests to test OAuth's well-known vulnerabilities"
    },
    "tests": [    
        {
            "test": {
                "name": "PKCE Downgrade",
                "description": "Tries to remove code_challenge parameter",
                "type": "active",
                "sessions": [
                    "s1"
                ],
                "operations": [
                    {
                        "session": "s1",
                        "action": "start"
                    },
                    {
                        "action": "intercept",
                        "from session": "s1",
                        "then": "forward",
                        "message type": "authorization request",
                        "preconditions": [
                            {
                                "in": "url",
                                "check param": "code_challenge",
                                "is present": true
                            }
                        ],
                        "message operations": [
                            {
                                "from": "url",
                                "remove parameter": "code_challenge"
                            }
                        ]
                    }
                ],
                "result": "incorrect flow s1"
            }
    
        }
    ]
}
\end{lstlisting}

The first Operation defined in this test at line 17 is an operation that is used to start the session (and the browser). The automated browser will execute a series of actions defined by the user in a session track. The actions in this case will do a complete login in a website that uses OAuth as SSO login option. During the execution of the actions, language's Operations will be executed. At line 21 there is an Operation used to intercept an "authorization request" message, that is defined in an apoosite file where all Message Types are defined. Once an authorization request message is intercepted, the preconditions at line 32 are executed, checking that the parameter we want to test the vulnerability is used. This is done because the parameter "code\_challenge" is not an optional parameter for the OAuth protocol, so, if it is not present, i want the test to result "not applicable" instead of failed or passed.
The next part of the example is the Message Operation at line 35, where I tell to remove from the intercepted message's url the "code\_challenge" parameter.
The last part of the test is the definition of the result, the result is part of the evaluation of a test, in this case it is set to "incorrect flow s1", this means that I want the test to be considered passed if the execution of the session s1 is incorrect that is, if the execution of the session s1 encounters an error or an unexpected page.


\subsection{Language structure}
\includegraphics[width=\textwidth]{language_structure.png}

\subsubsection{Test suite}
The test suite is the main component which contains all the other one, it is composed by:
\begin{itemize}
    \item Test suite name, the name of the test suite
    \item Test suite description, the description of the test suite
    \item Tests, which is a list containing the tests to be executed
\end{itemize}

\subsubsection{Test}
The Test object is the one that actually defines a test. As said earlier, a test is contained in a Test Suite, and has various items:
\begin{itemize}
    \item name
    \item description
    \item type, it can be "active" or "passive"
    \item sessions, which is a list of the sessions which are needed in this test
    \item result, (only for actives) it defines the conditions over which the test is considered passed or not.
    \item operations, a list of operation objects which will be executed in the Test object
\end{itemize}

it can be defined either as an active or a passive test, depending on the type of actions it has to do on the intercepted messages. If a test doesn't need to manipulate the flow or the content of the messages, then it is considered passive, otherwise it is considered active.
The list of Operations contained in a Test is executed iteratively one after the other.

\subsubsection{Operation}
The operation object is the thing that define what a test actually does. As shown in the image above, an operation could be either a standard operation or a session config operation, the latter is used to manage the sessions for the active tests (i.e. start, stop, pause). Depending on the type of test which an Operation is defined into, the standard Operation can be active or passive.
In both cases, an operation has to contain the \textbf{message type} which defines the type of message to be intercepted in that particular operation (more info in the dedicated paragraph).
\\A \textbf{passive} operation has as objective to verify the presence (or absence) of some text or parameters in the intercepted message, it should contain one of the following options:
\begin{itemize}
    \item A list of Check objects, which are then executed to check the presence of some text or parameter
    \item A regex inspection, which executes a inspection considering the intercepted message as plain text and executing a regex over it, if the regex has a match, the operation is considered passed, otherwise failed. Note that when a regex is used, it has to be specified also the message section over which to be executed (boy,head, url)
\end{itemize}

If the Test where the operations are defined is an \textbf{active} test, so if the intercepted messages need to be manipulated in some way, an active Operation has to be defined. It is composed by:
\begin{itemize}
    \item action, the action it has to do (intercept, validate)
    \item from session, from which session to expect the message to be intercepted
    \item then, the action to to after the receiving and manipulation of the message (forward or drop)
    \item replace request (or response), specify a previously saved message in order to replace it to the intercepted one
    \item preconditions, a list of Precondition objects
    \item message operations, a list of Message Operation objects, which will do the actual manipulation of the intercepted message
\end{itemize}

If the action is set to "\textbf{validate}" the operation becomes like a passive operation, because its objective is just to verify that some messages are as expected. It will contain or a regex or a list of checks to be done.

\subsubsection{Message Operation}
The message operation is the Object that actually does the manipulations on the intercepted messages. It is composed by:
\begin{itemize}
    \item from, the message section to work on
    \item decode parameter (optional) it indicates which parameter or string to be decoded before processed
    \item encodings (optional) the list of encodings to be applied to the parameter or text to be decoded. The supported encodings are base64, deflate, url
    \item remove match word (optional), remove text from te specified section in the matched message, it uses a regex
    \item edit, edit the matched text
    \item save, (optional) used to save an entire message in a variable in a way it can be used in future operations
    \item add, (optional) add some text after the matched text
    \item type (optional) specify the type of edit you want to do over a decoded parameter
\end{itemize}

In a message operation there is the possibility to specify a parameter or some text to be decoded before manipulation, to do that specify with "decode parameter" the parameter to be decoded and with "encodings" the encodings necessary to decode the parameter. The parameter (or text) decoded, at the end of the Message operation will be encoded again automatically.
The decoded parameter can be manipulated by means of the "\textbf{type}" tag, there is the possibility to intepreter the decoded parameter as plain text, and to edit it using some actions:
\begin{itemize}
    \item txt remove
    \item txt edit
    \item txt add
    \item txt save
\end{itemize}
All the previous tags accept a regex, and whatever that regex matches will be edited or added or saved.

Another possibility is to interpeter the decoded text as xml, assigning the type tag "xml".
This way we have various possible operations to be done on the xml:
\begin{itemize}
    \item remove tag
    \item remove attribute
    \item edit tag
    \item edit attribute
    \item add tag
    \item add attribute 
    \item save tag
    \item save attribute
\end{itemize}

\subsection{Message type definition}
The message type definition is needed in order to define some types of message that will be later used in the language to intercept them.
The message type definition is not actually part of the language, but it is stored in a file in the burp folder. Anyway, the definition of the type of messages uses the same Objects as the language.
A message type object is defined using these tags:
\begin{itemize}
    \item name, the name that will be used in the language to reffer to this message type
    \item is request, se to true if the searched message is a request, false otherwise
    \item response name, the name that will be used in the language to reffer to the response of the searched message
    \item checks, a list of Check objects used to identify the message. If evaluated to true, the message is considered found
\end{itemize}
This is an example that defines the saml request and the saml response messages

\begin{lstlisting}[language=json]
{
    "message_types": [
        {
            "name": "saml request",
            "is request": true,
            "checks": [
                {
                    "in": "url",
                    "check param": "SAMLRequest",
                    "is present": true
                }
            ]
        },
        {
            "name": "saml response",
            "is request": true,
            "checks": [
                {
                    "in": "body",
                    "check param": "SAMLResponse",
                    "is present": true
                }
            ]
        }
    ]
}
\end{lstlisting}
So, if "saml request" is used in an Operation, the message having the parameter SAMLRequest in his url will be intercepted an processed by the Operation.


\section{The oracle}
The ensemble of all parts of the language that decide the result of the tests is called Oracle,
the oracle decides whether a test should be considered passed or failed. I decided to build the oracle in a way that can be almost fully customized by the user. It is based on three main components:
\begin{itemize}
    \item Evaluation of the complete (or incomplete) execution of the session track 
    \item Evaluation of the Precondition objects
    \item Evaluation of the Validate objects
\end{itemize}
If all of the above conditions are met, the test is considered passed, otherwise it is considered failed.
The oracle can be built for example using Validate objects verifying that some intercepted messages satisfy some conditions like having a particular parameter or string in them.
To identify abnormal pages like error pages the session track evaluation should be sufficient, because if some of the actions could not be executed means that the original "flow" of pages was not followed.

\section{Sessions}
A session is defined by a session track, which is a series of commands that the browser will execute automatically during execution of the Tests. There is the possibility of defining and using more than one session, in a way that (i.e.) reply tests can be executed.
As said in the previous sections, a "from session" tag can be specified in the Operation, this will tell in which of the available session search the desired message. To define the session track I have taken inspiration from the one used in the Micro Id Gym tool \cite{claudio_grisenti}\cite{stefano_faccini}, adding some options like "wait" and "clear cookies" functionalities.
An example of a session track:

\begin{lstlisting}[]
    open | https://www.google.com/ |
    click | id=L2AGLb |
    click | link=Accedi |
    click | id=identifierId |
    type | id=identifierId | matteo.bitussi@studenti.unitn.it
    click | id=identifierNext |
    click | id=clid |
    type | id=clid | matteo.bitussi@unitn.it
    click | id=inputPassword |
    type | id=inputPassword | password
    click | id=btnAccedi |
    click | link=Gmail |
\end{lstlisting}

This session track will do the login on the Unitn website using some credentials and password. The actions supported are:
\begin{itemize}
    \item open | url |, to open an url
    \item click | id=, link=, xpath= |, to click on a http object with the given id, link or xpath
    \item type | id= | text, to write on a given http element the given text
    \item wait | milliseconds, to make the execution of the session wait for a given time 
    \item clear cookies |, to make the browser of the session clear all of the cookies in it
\end{itemize}




\chapter{Implementation}
In this chapter I will describe the implementation of the language and the plugin, and also the problems faced and the solutions adopted.

\section{The plugin}
For the implementation of the Burp's plugin I have decided to start from a work done by Wendy Barreto \cite{wendy_barreto}, which did a similar plugin for OIDC and OAuth SSO protocols, this was a good base to start with my implementation. The plugin code is written in Java, I used the Burp's interface classes to interact with it.
The standard usage of Burp Suite is based on the execution of a browser which connects to the Burp's proxy, in a way that all the packets can be intercepted, viewed or edited and forwared or dropped from the Burp interface. The tester would do some actions on the browser and watch the flowing packets in Burp and then check them or edit them. With the plugin the idea is the same, but the operation done on the browser and the checks or edits on the messages are made automatically, in a way that the tester doesn't have to do them by itself.

\subsection{Interface}
* insert image *

\subsection{Test execution}
The test execution differs from static to dinamic, as static tests don't need the the edit of the messages, the execution of the \gls{session track} is done once, the messages are saved and the tests are executed on the saved messages. I have also added the possibility of exporting the saved messages to a file, in a way that they can be imported in the plugin and tested again.
On the other hand, active tests needs to edit the messages, so the execution of the track has to be repeated for each test.

\subsection{Decoding \& encoding of parameters}
As said in the previous chapter, the encoding and decoding of parameters is possible. To do that, a list of encodings has to be provided, i.e. url, base64, deflate. Once the specified message is intercepted, the parameter is taken and decoded following the order of the provided encodings. To do that, i used part of the code of SAML Raider \cite{saml_raider} which did the decoding of SAML Requests and responses parameters. I've taken that part of the code and edited it to fit the plugin. SAML Raider is a Burp's plugin used to manage SAML certificates.

\subsection{SAML certificate managing}
In SAML Requests and responses there is sometime the need to remove or edit the certificate associated to that request or response, so, to speed up the process I did add a specific tag in the language to remove or edit the certificate signature. There is still the possibility of doing it by editing the SAML request or response with a regex, but this way is more convinient.
To do this, i used a part of the code of SAML Raider \cite{saml_raider}, editing it to fit my needs.

\subsection{Oracle}

To identify abnormal pages like error pages the \gls{session track} evaluation should be sufficient, because if some of the actions could not be executed means that the original "flow" of pages was not followed.

\subsection{Session managing}
The sessions are managed independently, each session is basically a browser that is launched when a session is started. Each session can follow a different \gls{session track} defined in the apposite tabs. Every session is ran in a separated thread to make parallelism possible. By the use of specific commands in the language, is possible to do some actions on each session, like stop it, pause it, or clear its cookies. Each browser uses a different proxy port, so that it is possible to know from which session the messages come from.





\chapter{Uses cases}
In this chapter I will talk about some examples of use cases in which my language could be used.

\section{SAML Use-Case}
My work has been used by Sofia Zanrosso, for her bachelor thesis, her objective was to search for SAML vulnerabilities and to define a series of well-known test to verify them.
She defined all the tests using my language and tested multiple SAML implementation with it.

\section{OAuth \& OIDC Use-Case}


\chapter{Related work}
In this chapter i will talk about other tools and works that are related to mine. I will then compare them to my language.

\section{SSO Testing language and Plugin}
This is a similar plugin and language done by Wendy Barreto \cite{wendy_barreto} to test OAuth and OIDC SSO protocols in a more dinamic way. Previously in this thesis i said that I have taken part of her work which was a good base to start with my work, editing it to suit my needs. My idea of a language that could be used for any type of test over HTTP was born when I used her plugin, which was limited to OAuth and OIDC tests. I wanted to enlarge the possible tests without a restiction on a specific protocol. One of the things that I used is the interface of her plugin, adding some buttons and tabs to deal with multiple session tracks and added functionalities.

\begin{center}
    \begin{tabular}{ |c|c|c| }
        \hline
        Action                                    & Old language         & New language                             \\
        \hline\hline
        Custom message filtering                  & Only on active tests & supported                                \\
        Edit string                               & only by regex        & supported with regex and check construct \\
        Remove string                             & only by regex        & supported with regex and check construct \\
        Add string                                & not supported        & supported                                \\
        Check parameter                           & only with regex      & with regex and check construct           \\
        Multiple operations in single message     & not supported        & supported                                \\
        Saving and reusing of values and messages & not supported        & supported                                \\
        Multiple sessions in single test          & not supported        & supported                                \\
        Custom oracle definition                  & not supported        & by using regex and checks                \\
        \hline
    \end{tabular}
\end{center}

\section{Micro ID Gym}
Another plugin for burp that was developed by two my colleagues was Micro ID Gym, a tool to test OIDC and OAuth implementations. This plugin was used by Wendy Barreto to do her work.
The old plugin of Stefano and Claudio was based on a track, which defined some actions to be done by the browser, which was a selenium istance. The plugin checks the messages and based on the test defined in the plugin tells if there is a vulnerability or not.
Starting from the first version of Stefano, to the last of Wendy, the plugin was improved. In the first two versions of Stefano and Claudio the plugin had its tests hard-coded, in a way that ony the supported tests could be executed, with little settings to change. If a new test had to be implemented, the plugin had to be recompiled. This version of the plugin worked well, but as said, the tests could not be customized or adapted by the user.

\section{Last plugin version}
Wendy improved the plugin by removing the staticity of the test, adding the possibility to customly define all of the tests with the use of a JSON language. \\
This is the last version of the plugin which i started working to. The plugin supported the definition of passive and active tests: passive tests are tests where the messages are not edited, active tests are tests where there could be an edit of one or more messages. The availabe test actions worked well, but there were some limitations on the possible actions, especially in the active tests. For example:
\begin{itemize}
    \item Limited oracle for the verification of active tests, having just the verification of the correct execution of the operation and a check for the string "error" on the last page of the browser
    \item The filtering of which message to check or edit for static tests was limited: ( only "Authorization grant message", "Response messages", "Request messages" and "All messages")
    \item Only regex were supported to search something in a message
    \item Unable to work over encoded parameters
    \item Impossibility of doing multiple operations on a single message
    \item Impossibility of saving a parameter and using it somewhere else
\end{itemize}
Some of which stated as future works in Wendy's thesis.
Moreover, the language was thought to be used with tests for OIDC and OAuth, other SSO protocols such as SAML could not be tested, because of the fact that SAML parameters are encoded, so editing them or verifying them is not possible. This is the biggest limitation that made me decide to redesign the language.


\chapter{Conclusions, Limitations and Future Works}
One of the biggest limitations that the tool has is the session track actions automations, it often happens that some captcha are encountered during execution, making impossible to proceed. Moreover, the track execution is limited, there is only a possible flow of actions (the one defined) and there is not the possibility of inserting if then else constructs that could help to differentiate the actions based on the actual page or popup. For example, it could happen that an "limited time offer" popup could appear in a website only in a particular time, the execution of the session track could be compromised by that, making impossible to distinguish wether the test failed because of the tested vulnerability or the actual popup.






\endgroup


% bibliografia in formato bibtex
%
% aggiunta del capitolo nell'indice
\addcontentsline{toc}{chapter}{Bibliografia}
% stile con ordinamento alfabetico in funzione degli autori
\bibliographystyle{plain}
\bibliography{biblio}
%%%%%%%%%%%%%%%%%%%%%%%%%%%%%%%%%%%%%%%%%%%%%%%%%%%%%%%%%%%%%%%%%%%%%%%%%%
%%%%%%%%%%%%%%%%%%%%%%%%%%%%%%%%%%%%%%%%%%%%%%%%%%%%%%%%%%%%%%%%%%%%%%%%%%
%% Nota
%%%%%%%%%%%%%%%%%%%%%%%%%%%%%%%%%%%%%%%%%%%%%%%%%%%%%%%%%%%%%%%%%%%%%%%%%%
%% Nella bibliografia devono essere riportati tutte le fonti consultate 
%% per lo svolgimento della tesi. La bibliografia deve essere redatta 
%% in ordine alfabetico sul cognome del primo autore. 
%% 
%% La forma della citazione bibliografica va inserita secondo la fonte utilizzata:
%% 
%% LIBRI
%% Cognome e iniziale del nome autore/autori, la data di edizione, titolo, casa editrice, eventuale numero dell’edizione. 
%% 
%% ARTICOLI DI RIVISTA
%% Cognome e iniziale del nome autore/autori, titolo articolo, titolo rivista, volume, numero, numero di pagine.
%% 
%% ARTICOLI DI CONFERENZA
%% Cognome e iniziale del nome autore/autori (anno), titolo articolo, titolo conferenza, luogo della conferenza (città e paese), date della conferenza, numero di pagine. 
%% 
%% SITOGRAFIA
%% La sitografia contiene un elenco di indirizzi Web consultati e disposti in ordine alfabetico. 
%% E’ necessario:
%%   Copiare la URL (l’indirizzo web) specifica della pagina consultata
%%   Se disponibile, indicare il cognome e nome dell’autore, il titolo ed eventuale sottotitolo del testo
%%   Se disponibile, inserire la data di ultima consultazione della risorsa (gg/mm/aaaa).    
%%%%%%%%%%%%%%%%%%%%%%%%%%%%%%%%%%%%%%%%%%%%%%%%%%%%%%%%%%%%%%%%%%%%%%%%%%
%%%%%%%%%%%%%%%%%%%%%%%%%%%%%%%%%%%%%%%%%%%%%%%%%%%%%%%%%%%%%%%%%%%%%%%%%%

\titleformat{\chapter}
{\normalfont\Huge\bfseries}{Allegato \thechapter}{1em}{}
% sezione Allegati - opzionale
\appendix
\chapter{PKCE test Example complete}
\label{chap:PKCE_complete}
\begin{lstlisting}[language=json]
{
    "test suite": {
        "name": "OAuth Active tests",
        "description": "A series of tests to test OAuth's well-known vulnerabilities",
        "filter messages": true
    },
    "tests": [     
        {
            "test": {
                "name": "PKCE Downgrade",
                "description": "Tries to remove code_challenge parameter",
                "type": "active",
                "sessions": [
                    "s1"
                ],
                "operations": [
                    {
                        "session": "s1",
                        "action": "start"
                    },
                    {
                        "action": "intercept",
                        "from session": "s1",
                        "then": "forward",
                        "message type": "authorization request",
                        "preconditions": [
                            {
                                "in": "url",
                                "check param": "code_challenge",
                                "is present": true
                            }
                        ],
                        "message operations": [
                            {
                                "from": "url",
                                "remove parameter": "code_challenge"
                            }
                        ]
                    }
                ],
                "result": "incorrect flow s1"
            }
        }
    ]
}
\end{lstlisting}

\chapter{Language comparison}
\label{attachment:languages_comparison}

\begin{center}
    \begin{tabular}{|l|l|l|}
        \hline
        Action                                    & Old language         & New language                             \\
        \hline\hline
        Custom message filtering                  & only on active tests & supported                                \\
        Edit string                               & only by regex        & supported with regex and check \\
        Remove string                             & only by regex        & supported with regex and check\\
        Add string                                & not supported        & supported                                \\
        Check parameter                           & only with regex      & with regex and check construct           \\
        Multiple operations in single message     & not supported        & supported                                \\
        Saving and reusing of values and messages & not supported        & supported                                \\
        Multiple sessions in single test          & not supported        & supported                                \\
        Custom oracle definition                  & not supported        & by using regex and checks                \\
        \hline
    \end{tabular}
\end{center}






\end{document}
