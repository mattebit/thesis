\chapter{Use cases}
\label{chap:Use_cases}
In this chapter some examples of use cases in which the work of this thesis has been used will be shown.

\section{SAML Use-Case}
During the last stages of the development and testing of the tool, a strict collaboration between my colleague Sofia Zanrosso and me has started. Her objective was to create a SAML Test Suite in order to facilitate automatic pentesting over SAML \cite{sofia_zanrosso}. The tool defined in this thesis was compared with other ones and was used to define and execute the tests. During the progress of both our works, a lot of feedbacks and bugs has been reported to me, speeding up the testing phase of the tool. At the same time, we found that my tool was in some respects better compared the other alternatives. For example:
\begin{itemize}
    \item Other plugins were giving false positives on multiple tests, while this was not happening in my tool.
    \item "the previously employed transition times between tools have been greatly reduced"
    \item "making it possible to analyze almost completely the vulnerabilities of the tested" subjects"
\end{itemize}


\section{OAuth \& OIDC Use-Case}    
The \Gls{OAuth} and \Gls{OIDC} tests defined and used in \cite{claudio_grisenti,wendy_barreto} has been re-defined in the SBTL language and has been used to evaluate the correctness of the tool with respect to the already existing ones. The tool discussed in this thesis and the other ones had the same results with the same tests. Due to the new available functionalities introduced with SBTL, new tests has been defined, especially active ones.





